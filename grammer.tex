\documentclass{article}
\usepackage{booktabs}
\usepackage{multirow}
\usepackage[utf8]{inputenc} % For UTF-8 encoding
\usepackage[greek]{babel} % For Greek language support
\usepackage{fontspec}
\usepackage[margin=1in]{geometry} % Adjust margins here
\usepackage{graphicx}

\setmainfont{Times New Roman}

\newcommand{\articles}[7]{
    \begin{table}[htbp]
        \centering
        \begin{tabular}{llcc}
            \toprule
                &        & singular & plural \\
                & gender &          &        \\
            \midrule
            \multirow{3}{*}{\rotatebox{90}{#1.}}
                & mas.    & #2        & #5 \\
                & fem.    & #3        & #6 \\
                & neu.    & #4        & #7 \\
            \bottomrule
        \end{tabular}
    \end{table}
}

\newcommand{\pronouns}[11]{
    \begin{table}[htbp]
        \centering
        \begin{tabular}{llcc}
            \toprule
                &        & singular & plural \\
                & person &          &        \\
            \midrule
            \multirow{3}{*}{\rotatebox{90}{#1.}}
                & I      & #2        & #7 \\
                & you    & #3        & #8 \\
                & he     & #4        & #9 \\
                & she    & #5        & #10 \\
                & it     & #6        & #11 \\

            \bottomrule
        \end{tabular}
    \end{table}
}

\begin{document}

\articles{nom}{ο}   {η}   {το}  {ιο}  {ιο}  {τα}
\articles{acc}{τον} {την} {το}  {τους}{τις} {τα}
\articles{gen}{του} {της} {του} {των} {των} {των}
\pronouns{nom}{εγώ}{εσύ}{αυτός}{αυτή}{αυτό}{εμείς}{εσείς}{αυτοί}{αυτές}{αυτά}

% \begin{table}[h]
% \centering
% \begin{tabular}{lccc}
% \toprule
%     Case       & Masculine & Feminine & Neuter \\
% \midrule
% \multicolumn{4}{c}{\textbf{Singular}}\\
% \midrule
%     Nominative & ο         & η        & το  \\
%     Accusative & τον       & την      & το  \\
%     Genitive   & του       & της      & του \\
% \midrule
% \multicolumn{4}{c}{\textbf{Plural}} \\
% \midrule
%     Nominative & οι        & οι       & τα  \\
%     Accusative & τους      & τις      & τα  \\
%     Genitive   & των       & των      & των \\
% \bottomrule
% \end{tabular}
% \caption{Greek Articles in Nominative, Accusative, and Genitive Cases}
% \end{table}

\begin{table}[h]
\centering
\begin{tabular}{lp{10cm}}
\toprule
\textbf{Case} & \textbf{Primary Uses} \\
\midrule
\textbf{Nominative} &
    \begin{itemize}
        \item Subject of a verb (e.g., \textit{Ο άντρας τρέχει.} - The man runs.)
        \item Predicate nominative (e.g., \textit{Αυτός είναι ο δάσκαλος.} - He is the teacher.)
    \end{itemize} \\
\midrule
\textbf{Accusative} &
    \begin{itemize}
        \item Direct object of a verb (e.g., \textit{Βλέπω τον σκύλο.} - I see the dog.)
        \item With certain prepositions (e.g., \textit{Πάω στην πόλη.} - I go to the city.)
        \item Time expressions (e.g., \textit{Έμεινα την νύχτα.} - I stayed the night.)
    \end{itemize} \\
\midrule
\textbf{Genitive} &
    \begin{itemize}
        \item Possession (e.g., \textit{Το βιβλίο του Γιάννη.} - The book of John.)
        \item Origin or source (e.g., \textit{Τα προϊόντα της Ελλάδας.} - The products of Greece.)
        \item Partitive genitive (e.g., \textit{Ένα κομμάτι του κέικ.} - A piece of the cake.)
        \item With certain verbs and prepositions.
        \item Description.
    \end{itemize} \\
\midrule
\textbf{Vocative} &
    \begin{itemize}
        \item Direct address (e.g., \textit{Γιάννη, έλα εδώ!} - John, come here!; \textit{Κύριε, μπορείτε να με βοηθήσετε;} - Sir, can you help me?)
    \end{itemize} \\
\bottomrule
\end{tabular}
\end{table}


\end{document}
